\newpage
\section{Use-cases}

The table below presents a range of potential use-cases for contracts that
utilize block timestamps and block numbers. It also provides a safety rating
for each use-case under both PoS and PoW frameworks. These contracts serve as
examples and focus solely on issues related to block values, excluding any
other types of vulnerabilities.

\newcolumntype{C}[1]{>{\centering\arraybackslash}p{#1}}
\newcolumntype{L}[1]{>{\raggedright\arraybackslash}p{#1}}

\begin{table}[H]
\centering
\small % Reduce the font size
\begin{tabularx}{\textwidth}{L{3.5cm}L{3.5cm}C{0.5cm}C{0.5cm}X} % Fixed width for the second column
\hline
\textbf{Contract Name}& \textbf{Contract Description}                                                                                                                           & \textbf{PoW} & \textbf{PoS} & \textbf{Reason}                                                                                                                                                                                \\ \hline
Firework              & A contract designed to activate New Year's fireworks precisely at 12 AM.                                                                & bad          & bad          & Precise timing cannot be achieved in either PoW or PoS systems.                                                                                                                                \\ \hline
FastWithdrawChallenge & The first individual to withdraw after a specified timestamp wins the money.                                                            & bad          & good         & In a PoW system, the block timestamp can be influenced by up to 15 seconds by a malicious miner.                                                                                               \\ \hline
BlockNumberTimeLock   & A contract that locks Ethereum for a set period, utilizing block numbers as a proxy for time, based on an assumed 14-second block time. & bad          & bad          & In both PoW and PoS, the block time does not consistently maintain a 14-second duration.                                                                                                       \\ \hline
BlockNumberLock       & A contract that locks Ethereum for a specified number of blocks without guaranteeing exact timing.                                      & good         & good         & In both PoS and PoW it is safe to use when it is clearly communicated to the user that a lock will be maintained for a specified number of blocks, without guaranteeing a specific time period


\end{tabularx}
\caption{Analysis of Various Smart Contracts}
\label{tab:smart_contracts}
\end{table}
%
%\subsection{Timelocks for ICO}
%https://blog.openzeppelin.com/bypassing-smart-contract-timelocks
%
%\subsection{Timelocks}
%A timelock is a smart contract that serves as a mechanism to delay function calls form another contract for
%a predefined amount of time. Timelocks find their primary application in governance systems,
%where they play a crucial role in implementing delayed administrative actions.
%Their presence often signifies the credibility of a project and underscores the commitment of its owners
%to its long-term success. \cite{timelock2021}.
%
%Openzeppelin provides such a timelock in their library (@openzeppelin/contracts/governance/TimelockController.sol).
%A part of this timelock code can be seen in Listing \ref{lst:timelock} \cite{timelock_code}.
%
%\lstinputlisting[language=Solidity, caption={Openzeppelin Timelock}, label={lst:timelock}]{../dev/contracts/timelock.sol}
%
%\subsection{State Machine}
%Contracts often act as a state machine. It's common to have multi-stage
%processes where certain actions are allowed only during specific phases.
%Timestamps can be used to enforce time-based conditions within a contract \cite{soliditydocs_statemaschine}. \\
%The following example shows a contract that implements the transition of stages (states) based on timestamps \cite{stagedcontract_code}.
%
%\begin{solidity}
%modifier timedTransitions() {
%    if (stage == Stages.AcceptingBlindBids && block.timestamp >= creationTime + 6 days) {
%        nextStage();
%    }
%    if (stage == Stages.RevealBids && block.timestamp >= creationTime + 10 days) {
%        nextStage();
%    }
%    _;
%}
%\end{solidity}

\section{Best practices under PoS}

\begin{itemize}
  \item Do not use block.number as a proxy for time.
  \item Do not use block.timestamp to trigger exact timings.
  \item Avoid using block.number in PoW as a proxy for time, because it is not guaranteed that the block time is fixed at 14 seconds \\
  \item Avoid using block.number in PoS as a proxy for time, because the block time can be a multiple of 12 seconds and may change in the future (instead use block.timestamp)\\
  \item Avoid using block.timestamp in PoW, because it can be manipulated by miners \\
  \item Avoid using block.timestamp in PoS for triggering events with an exact time (more accurate than 12 seconds, e.g. silvester) \\
  \item It is safe to use block.timestamp in PoS, if the time-dependent event must only be higher than a certain timestamp (e.g. timelocks) \\
  \item Never use block.timestamp in a comparison (do not do: e.g. block.timestamp == 1702446773), instead use $>$ or $>=$
\end{itemize}


