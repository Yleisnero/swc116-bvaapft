\section{Tools and code properties}

\subsection{conkas}
\begin{lstlisting}[language=bash, caption="conkas output for the time lock contract"]
    Analysing /contracts/time_lock.sol:TimeLock...
    Vulnerability: Transaction Ordering Dependence. Maybe in function: withdraw(). PC: 0x3b9. Line number: 29.
    Vulnerability: Integer Overflow. Maybe in function: withdraw(). PC: 0x3d3. Line number: .
    If a = 115792089237316195423570985008687907853269984665640564039457584007913129639920
    and b = 63
\end{lstlisting}

\subsection{mythril}
Mythril looks for the opcodes TIMESTAMP and NUMBER in the code and warns the user if they are found \cite{mythril_opcodes}. \newline
\verb|predictable_ops = ["COINBASE", "GASLIMIT", "TIMESTAMP", "NUMBER"]|

\begin{lstlisting}[language=bash, caption="Mythril output for the time lock contract"]
...

==== Dependence on predictable environment variable ====
SWC ID: 120
Severity: Low
Contract: TimeLock
Function name: withdraw()
PC address: 697
Estimated Gas Usage: 1985 - 2460
A control flow decision is made based on The block.number environment variable.
The block.number environment variable is used to determine a control flow decision. Note that the values of variables like coinbase, gaslimit, block number and timestamp are predictable and can be manipulated by a malicious miner. Also keep in mind that attackers know hashes of earlier blocks. Don't use any of those environment variables as sources of randomness and be aware that use of these variables introduces a certain level of trust into miners.
--------------------
In file: /contracts/time_lock.sol:25

require(block.number >= users[msg.sender].unlockBlock, 'lock period not over')

...
\end{lstlisting}