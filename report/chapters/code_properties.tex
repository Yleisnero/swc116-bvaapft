\section{Tools and code properties}
Different tools exist, which analyze smart contracts and trying to find possible weaknesses in the contracts. \newline
In the following are listed five tools and how they detect "block values as a proxy for time" in the code.

\subsection{Conkas}
Conkas checks if a variable in the code is called 'timestamp' \cite{conkas_opcodes}
\begin{lstlisting}
    TIME_MANIPULATION_VAR = 'timestamp'
    def __is_based_on_time(data):
    for var in get_vars(data):
        var_name = str(var)
        if var_name == TIME_MANIPULATION_VAR:
            return True
        return False
\end{lstlisting}

\begin{lstlisting}[language=bash, caption="conkas output for the timed\_crowdsale.sol contract"]
==== Dependence on predictable environment variable ====
SWC ID: 116
Severity: Low
Contract: TimedCrowdsale
Function name: run()
PC address: 98
Estimated Gas Usage: 176 - 271
A control flow decision is made based on The block.timestamp environment variable.
The block.timestamp environment variable is used to determine a control flow decision. Note that the values of variables like coinbase, gaslimit, block number and timestamp are predictable and can be manipulated by a malicious miner. Also keep in mind that attackers know hashes of earlier blocks. Don't use any of those environment variables as sources of randomness and be aware that use of these variables introduces a certain level of trust into miners.
--------------------
In file: /contracts/timed_crowdsale.sol:14

if (isSaleFinished()) {
        emit Finished();
    } else {
        emit notFinished();
    }
\end{lstlisting}

\subsection{Mythril}
Mythril looks for the opcodes TIMESTAMP and NUMBER in the code and warns the user if they are found \cite{mythril_opcodes}. \newline
\verb|predictable_ops = ["COINBASE", "GASLIMIT", "TIMESTAMP", "NUMBER"]|

\begin{lstlisting}[language=bash, caption="Mythril output for the time lock contract"]
==== Dependence on predictable environment variable ====
SWC ID: 120
Severity: Low
Contract: TimeLock
Function name: withdraw()
PC address: 697
Estimated Gas Usage: 1985 - 2460
A control flow decision is made based on The block.number environment variable.
The block.number environment variable is used to determine a control flow decision. Note that the values of variables like coinbase, gaslimit, block number and timestamp are predictable and can be manipulated by a malicious miner. Also keep in mind that attackers know hashes of earlier blocks. Don't use any of those environment variables as sources of randomness and be aware that use of these variables introduces a certain level of trust into miners.
--------------------
In file: /contracts/time_lock.sol:25

require(block.number >= users[msg.sender].unlockBlock, 'lock period not over')
\end{lstlisting}

\subsection{Osiris}
Like Mythril, Osiris looks for the opcodes TIMESTAMP and NUMBER in the code and warns the user if they are found \cite{osiris_opcodes}. \newline
\begin{lstlisting}
    opcodes = {
        ...
        "BLOCKHASH": [0x40, 1, 1],
        "COINBASE": [0x41, 0, 1],
        "TIMESTAMP": [0x42, 0, 1],
        "NUMBER": [0x43, 0, 1]
        ...
    }
\end{lstlisting}

\begin{lstlisting}[language=bash, caption="Osiris output for the time lock contract"]
    TODO
\end{lstlisting}

\subsection{Oyente}
Because Osiris is built on top of Oyente, the same code for looking for opcodes is found in both tools \cite{oyente_opcodes}. \newline

\begin{lstlisting}
    opcodes = {
        ...
        "BLOCKHASH": [0x40, 1, 1],
        "COINBASE": [0x41, 0, 1],
        "TIMESTAMP": [0x42, 0, 1],
        "NUMBER": [0x43, 0, 1]
        ...
    }
\end{lstlisting}

\begin{lstlisting}[language=bash, caption="Oyente output for the time lock contract"]
    TODO
\end{lstlisting}

\subsection{SmartCheck}
SmartCheck uses pattern matching on the .sol files to find usages of block.timestamp inside expression which contain "==" or "!=" and which are not a function argument \cite{smartcheck}.
\begin{lstlisting}[language=xml]
    <XPath>
    //expression
        [comparison]
        [expression//environmentalVariable
            [matches(text()[1], "^block\.timestamp|now$")]
            [not(ancestor::*[4][self::functionCall])]]
    </XPath>
\end{lstlisting}

\begin{lstlisting}[language=bash, caption="Oyente output for the time lock contract"]
    TODO
\end{lstlisting}