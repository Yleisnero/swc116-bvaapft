\section{Block number in Ethereum}
\subsection{Proof-of-Work Block Number}
The yellow paper of Ethereum states that each block has
an integer value as a block number. The block number
of a specific block must be exactly one unit higher than the block
number of the previous block \cite{ethyellowpaper2023}.
Smart contract developers attempted to utilize the block number to calculate
the current time, operating under the assumption that the time interval between
two blocks always remains constant at fourteen seconds. Theoretically, one could
calculate time differences between blocks by multiplying the block numbers by fourteen 
and subtracting them from each other.
However, this approach proves impractical in reality due to the variable
intervals between blocks, as explained earlier. Moreover, these block intervals
can change unexpectedly, for instance, when the difficulty bomb was introduced
or during a chain fork \cite{swc116}.
As a result the block number should not be used as a proxy for time.
%An example for this misusage can be seen in the Listing \ref{lst:number_weakness}.
%There the block number is used to lock a function of a contract for a specific time.

%\lstinputlisting[language=Solidity, caption={Block Number Weakness in PoW},linerange={4-17}, label={lst:number_weakness}]{../dev/contracts/blocknumbertimelock.sol}

\subsection{Proof-of-Stake Block Number}
In PoS, just like in PoW, the block number increases by one with each new
block. In PoS, the block intervals are for 99.05\% at 12 seconds. Consequently,
it might appear that using the block number as a time proxy in PoS is a viable
option. However, it's not advisable, because validators go offline from time to
time. This causes the block time to increase and therefore leads to an
inaccuracy of the calculation of time based on the block number. \\
% (like the one in Listing \ref{lst:number_weakness})
A contract initially deployed under PoW would
become impractical or malfunction in a PoS environment due to the inherent differences in the block times.

\subsection{Conclusion}
The transition from PoW to PoS has significantly stabilized block times,
particularly over short periods. However, it's important to note that using
block.number as a proxy for time is still not advisable, as the fixed value
with 
12 seconds per slot might change in the future. In earlier PoS designs, the
slot time was set at 6 seconds \cite{block_time_6_to_12_sec}. There have
also been proposals to adjust the slot time to 8 seconds
\cite{proposed_block_time_8_seconds}. This indicates that for shorter
durations, block times under PoS are markedly more stable than those under
PoW. Conversely, for longer timeframes, this stability is less assured due
to the potential for changes in slot time.


