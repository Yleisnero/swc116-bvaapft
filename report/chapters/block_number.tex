%\section{Block number}

\subsection{Block Number}

The yellow paper of Ethereum states the each block has a positive integer value
as a block number. The number of a block needs to be higher then the block
number of the previous block \cite{ethyellowpaper2023}.
\begin{equation} \label{eq:blocknumber}
    H_i \equiv P(H)_{H_i} + 1
\end{equation}
This means that the number of a block is a reliable source to find out the
number of the current block and in which block a transaction was included
\cite{eth_blocks}.

\paragraph{Software Implementation}
In proof of work based systems block times are probabilistic, because they
depend on the mining difficulty. The mining difficulty in those systems can
vary and therefore the block time can also vary \cite{eth_blocks}. \newline In
contrast proof of stake based systems have a relatively constant block time.
Before the Paris update of Ethereum, Ethereum was based on proof of stake. The
so called Merge introduced proof of stake for Ethereum on the 15.09.2022
\cite{eth_history}. The current version of Ethereum has a hardcoded block time
of 12 seconds \cite{eth_blocks}. \newline

\paragraph{Issues}
A weakness in smart contracts can happen, if developers try to use the block
number as a proxy for time. By assuming that the time between two blocks is
always fourteen seconds (in proof of work), it is theoretically possible to
calculate the current time, by dividing the block number by fourteen. In
practice this does not work, because the intervals between blocks can vary. On
top of that it can happen, that the block intervals change, e.g. when the
difficulty bomb was introduced or when a fork of the chain is done
\cite{swc116}. \newline In proof of stake based systems the block time is more
constant, which means the block number becomes a more reliable source for time
\cite{bellatrix_specs}. Nevertheless, the block number should not be used as a
proxy for time, because it is not guaranteed that the block time will always
stay the same in the future.	

\subsection{Weakness Example}

\lstinputlisting[language=Solidity, caption={Block Number Weakness},linerange={4-17}, label={lst:number_weakness}]{../dev/contracts/bad-blocknumber.sol}


% \subsubsection{Example without a weakness}
% \begin{solidity}
% contract Game {
%   uint startingBlock;
% 
%   constructor() public {
%     // Allowed to play afer 10 blocks
%     startingBlock = block.number + 10;
%   }
% 
%   fun play() public {
%     require(block.number >= startingBlock);
%   }
% }
% \end{solidity}


\subsection{Consequences}
User may experience inaccurate time behavior in the contract. Users may not
have access their Ethereum or other information stored in the contract.
