\section{Block timestamp}

\subsection{Background} 
The miner is responsible for setting the timestamp within a block. The yellow
paper defines which rules this value has to follow. But depending on the
specific consensus mechanism there are also consensous specification and
software imposed restrictions for the timestamp value. 

\paragraph{Yellow Paper}
Pre-merge the only restriction was the yellow paper, which set that the that
the timestamp of the new block has to be greater than the timestamp of the
parent block \cite{ethyellowpaper2023}.

\paragraph{Software Implementation - Proof-of-work}
The currently most used ethereum implementation Geth rejects blocks which
timestamps are more than 15 seconds in the future
\cite{go-ethereum-15-sek-limit}. Hence when the proof-of-work consensus
mechanism was used it was possible for the miner to tamper
with the timestamp and change it up to plus 15 seconds.

\paragraph{Software Implementation - Proof-of-stake}
Proof-of-stake ethereum introduced slots, where each slot has a time frame of
12 seconds \cite{seconds-per-slot-mainnet}\cite{seconds-per-slot-mainnet-doc}.
For each slot one random validator is selected to propose a block, which other
random validatores have vote about its validity. The specification defines
how to calculate the timestamp at a specific slot with the function
\textit{compute\_timestmap\_at\_slot} \cite{compute-timestamp-at-slot}. The
timestamp is calculated with $genesis\_time + slots\_since\_genesis *
SECONDS\_PER\_SLOT$. The function \textit{process\_execution\_payload} then
checks if the blocks timestamp equals the calculated timestamp at that slot
\cite{process-execution-payload}. Hence each slot has a predefined timestamp
and it is therefore impossible to tamper with but also much more easier to
predict.
% \begin{lstlisting}[language=python, caption=proof-of-stake consensous
% specification\cite{compute-timestamp-at-slot}]
% def compute_timestamp_at_slot(state: beaconstate, slot: slot) -> uint64:
%     slots_since_genesis = slot - genesis_slot
%     return uint64(state.genesis_time + slots_since_genesis * seconds_per_slot)
% \end{lstlisting}



% \paragraph{Yellow Paper}
% Miners set the block timestamp during the mining process (source), in
% that process miners have to follow only the restriction of the Ethereum
% Yellow-Paper, which is the following:
% 
% We define $H_s$ is the timestamp of Block $H$, $P(H)$ is the parent
% block of block $H$. The Yellow-Paper defines the following relation for a valid
% block timestamp \cite{ethyellowpaper2023}.
% 
% This definition essentially means the timestamp of the block $H_s$ in
% \ref{eq:1} must be greater then the timestamp of the previous (parent) block
% $P(H)_{H_s}$ \cite{ethyellowpaper2023}.

%\paragraph{Software Implementation}
%
%\begin{lstlisting}[language=go, caption=The restriction for the timestamp in Geth. Source: \textit{consensus/ethash/consensus.go} \cite{timestamp_code}]
%func (ethash *Ethash) verifyHeader(chain consensus.ChainHeaderReader, header, parent *types.Header, uncle bool, unixNow int64) error {
%  ...
%    // Verify the header's timestamp
%        if header.Time > uint64(unixNow+allowedFutureBlockTimeSeconds) {
%            return consensus.ErrFutureBlock
%        }
%    if header.Time <= parent.Time {
%        return errOlderBlockTime
%    }
%  ...
%}
%\end{lstlisting}
%
%\paragraph{Issues}

\begin{itemize}
\item Do not use block.timestamp as time locks, since developer should follow the yellow paper, it is not sure that software implements the restrictions.
\item Do not use block.timestamp as a source of randomness.
\end{itemize}

\subsection{Examples}
\subsubsection{Example containing weakness (Proof of work)}
\begin{solidity}
contract Firework {
    function startFirework() public {
        // 01.01.2024 00:00:00 GMT+0100
        require(block.timestamp > 1704063600);
    }
}
\end{solidity}

\subsubsection{Example without a weakness (Proof of work)}
\begin{solidity}
contract Game {
    uint expiry;

    constructor(uint expiryTimestamp) public {
        expiry = expiryTimestamp;
    }

    function play() public {
        // Safe to use because block timestamp can not be modified backwards
        require(block.timestamp < expiry);
    }
}
\end{solidity}
    
\subsection{Consequences}
A miner is gaining an unfair advantage if he is able to set the timestamp
of his mined block into the future, such that he is able to access
resources earlier then other users or trigger events before they are meant to happen.
