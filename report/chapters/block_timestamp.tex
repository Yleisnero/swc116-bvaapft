\section{Block timestamp}
\subsection{Proof-of-Work Block Timestamp}
As discussed in \ref{diff_adjustment} under PoW the block time had the
potential to vary from one block to another. Furthermore, the only restriction
miners had to follow when setting the block timestamp value was outlined in the
yellow paper, which regulated that the timestamp of the new block must be
greater than the timestamp of the parent block \cite{ethyellowpaper2023}.
Consequently, miners could gain an unfair advantage by manipulating the
timestamp of their mined block \cite{swc116}.

To prevent miners from setting the timestamp too far into the future, some of
the most widely used Ethereum implementations, such as Geth (go-ethereum),
implemented a rejection mechanism for blocks with timestamps exceeding 15
seconds into the future \cite{go-ethereum-15-sek-limit}.

Hence, when the PoW consensus mechanism was in use, miners had the ability to
manipulate the timestamp and adjust it by up to 15 seconds. This behavior led
to the establishment of the "15-second Rule"
\footnote{\url{https://consensys.github.io/smart-contract-best-practices/development-recommendations/solidity-specific/timestamp-dependence/}},
as a security best practice. The
rule suggests that the usage of the block timestamp is considered
safe if the time-dependent event can tolerate a variation of up to 15 seconds.

\subsection{Proof-of-Stake Block Timestamp}

The PoS Ethereum consensus specification provides instructions on how to
calculate the timestamp at a specific slot using the
\textit{compute\_timestamp\_at\_slot} function \cite{compute-timestamp-at-slot}.
This calculation is based on the following formula:

\begin{equation}
GENESIS\_TIME + slots\_since\_genesis *
SECONDS\_PER\_SLOT
\end{equation}


As dicuessed in \ref{pos_block_time} the Ethereum mainnet configuration, uses
value of 12 seconds for \textit{SECONDS\_PER\_SLOT}
\cite{seconds-per-slot-mainnet} \cite{seconds-per-slot-mainnet-doc}.
The mainnet beacon chain's value for \textit{GENESIS\_TIME} is $1606824023$, which
corresponds to the timestamp when the original PoS beacon chain was launched on
December 1, 2020.

Each validator calculates the block timestamp in the same way and verifies if
the timestamp in the block matches the calculated timestamp. If there is a
mismatch, the block is rejected \cite{process-execution-payload}.

\subsection{Conclusion}

Under PoW miners were able to manipulate timestamps for up to 15 seconds, which
could be used to gain an unfair advantage. Since PoS each slot, and
consequently each block, has a predetermined timestamp, it is
therefore impossible to tamper with, but it also becomes more predictable (very
bad for randomness).

