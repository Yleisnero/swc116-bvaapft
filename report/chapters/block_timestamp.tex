\section{Block timestamp}

\subsection{Background}


\paragraph{Yellow Paper}

Miners set the block timestamp during during the mining process (source), in
that process miners have to follow only the restriction of the Ethereum
Yellow-Paper, which is the following:

We define $H_s$ is the timestamp of Block $H$, $P(H)$ is the parent
block of block $H$. The Yellow-Paper defines the following relation for a valid
block timestamp \cite{ethyellowpaper2023}.

\begin{equation} \label{eq:1}
H_s > P(H)_{H_s}
\end{equation}

This definition essentially means the timestamp of the block $H_s$ in
\ref{eq:1} must be greater then the timestamp of the previous (parent) block
$P(H)_{H_s}$.

\paragraph{Software Implementation}

In the implementations of Ethereum, like Geth and Parity there is a
restriction. The timestamp of the new block must at most 15 seconds in the
future \cite{Conkas2021}. \newline

TODO: add sources to the code

\paragraph{Issues}

\begin{itemize}
\item Do not use block.timestamp as time locks, since developer should follow the yellow paper, it is not sure that software implements the restrictions.
\item Do not use block.timestamp as a source of randomness.
\end{itemize}


\paragraph{Do }

\subsection{Examples} 
\begin{lstlisting}[language=go, caption="The restriction for the timestamp in Geth. Source: consensus/ethash/consensus.go \cite{timestamp_code}"]
    func (ethash *Ethash) verifyHeader(chain consensus.ChainHeaderReader, header, parent *types.Header, uncle bool, unixNow int64) error {
      ...
        // Verify the header's timestamp
            if header.Time > uint64(unixNow+allowedFutureBlockTimeSeconds) {
                return consensus.ErrFutureBlock
            }
        if header.Time <= parent.Time {
            return errOlderBlockTime
        }
      ...
    }
    \end{lstlisting}
Source: SWC116 \cite{swc116}