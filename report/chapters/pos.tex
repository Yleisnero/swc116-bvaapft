\section{Introduction}

Smart contracts often utilize time values to trigger specific actions. The
values that can be employed for this purpose include the block timestamp and
the block number. However, under the Proof-of-Work consensus mechanism in
Ethereum, using these values has been associated with various security issues
\cite{swc116} \cite{Conkas2021} \cite{DASP2018} \cite{Osiris2018}
\cite{Oyente2016}.

Since the transition to the Proof-of-Stake consensus mechanism in
Ethereum, significant changes have occurred in how the timestamp value has to
be set and in the stability of the block time. This paper aims to discuss these
changes.

\section{Ethereum's Consensus Mechanisms}
Ethereum initially adopted the Proof-of-Work (PoW) consensus mechanism, a process known
for its security and decentralization but also criticized for its high energy
consumption and potential for centralization due to mining pools. PoW requires
miners to solve complex computational problems to validate transactions and
create new blocks, a process that consumes substantial computational power and
energy \cite{eth_pow}.

In response to these concerns, Ethereum transitioned to the Proof-of-Stake
(PoS) mechanism with "The Merge" on September 15, 2022. PoS, in contrast to
PoW, selects validators to create new blocks based on the number of coins they
hold and are willing to 'stake' as collateral. This approach significantly
reduces the energy requirements and aims to offer a more scalable and
environmentally friendly alternative \cite{eth_merge}.



\subsection{Conclusion}


\section{Block timestamp in Ethereum}
\subsection{Proof-of-Work Block Timestamp}
As discuessed in \ref{diff_adjustment} under PoW the block time had the
potential to vary from one block to another. Furthermore, the only restriction
miners had to follow when setting the block timestamp value was outlined in the
yellow paper, which regulated that the timestamp of the new block must be
greater than the timestamp of the parent block \cite{ethyellowpaper2023}.
Consequently, miners could gain an unfair advantage by manipulating the
timestamp of their mined block \cite{swc116}.

To prevent miners from setting the timestamp too far into the future, some of
the most widely used Ethereum implementations, such as Geth (go-ethereum),
implemented a rejection mechanism for blocks with timestamps exceeding 15
seconds into the future \cite{go-ethereum-15-sek-limit}.

Hence, when the PoW consensus mechanism was in use, miners had the ability to
manipulate the timestamp and adjust it by up to 15 seconds. This behavior led
to the establishment of the "15-second Rule"
\footnote{\url{https://consensys.github.io/smart-contract-best-practices/development-recommendations/solidity-specific/timestamp-dependence/}},
as a security best practice. The
rule suggests that the usage of the block timestamp is considered
safe if the time-dependent event can tolerate a variation of up to 15 seconds.

\subsection{Proof-of-Stake Block Timestamp}

The PoS Ethereum consensus specification provides instructions on how to
calculate the timestamp at a specific slot using the
\textit{compute\_timestamp\_at\_slot} function \cite{compute-timestamp-at-slot}.
This calculation is based on the following formula:

\begin{equation}
GENESIS\_TIME + slots\_since\_genesis *
SECONDS\_PER\_SLOT
\end{equation}


As dicuessed in \ref{pos_block_time} the Ethereum mainnet configuration, uses
value of 12 seconds for \textit{SECONDS\_PER\_SLOT}
\cite{seconds-per-slot-mainnet} \cite{seconds-per-slot-mainnet-doc}.
The mainnet beacon chain's value for \textit{GENESIS\_TIME} is $1606824023$, which
corresponds to the timestamp when the original PoS beacon chain was launched on
December 1, 2020.

Each validator calculates the block timestamp in the same way and verifies if
the timestamp in the block matches the calculated timestamp. If there is a
mismatch, the block is rejected \cite{process-execution-payload}.

\subsection{Conclusion}

Under PoW miners were able to manipulate timestamps for up to 15 seconds, which
could be used to gain an unfair advantage. Since PoS each slot, and
consequently each block, has a predetermined timestamp, it is
therefore impossible to tamper with, but it also becomes more predictable (very
bad for randomness).

\section{Block number in Ethereum}
\subsection{Proof-of-Work Block Number}
The yellow paper of Ethereum states that each block has
an integer value as a block number. The block number
of a specific block must be exactly one unit higher than the block
number of the previous block \cite{ethyellowpaper2023}.
Smart contract developers attempted to utilize the block number to calculate
the current time, operating under the assumption that the time interval between
two blocks always remains constant at fourteen seconds. Theoretically, one could
calculate time differences between blocks by multiplying the block numbers by fourteen 
and subtracting them from each other.
However, this approach proves impractical in reality due to the variable
intervals between blocks, as explained earlier. Moreover, these block intervals
can change unexpectedly, for instance, when the difficulty bomb was introduced
or during a chain fork \cite{swc116}.
As a result the block number should not be used as a proxy for time.
%An example for this misusage can be seen in the Listing \ref{lst:number_weakness}.
%There the block number is used to lock a function of a contract for a specific time.

%\lstinputlisting[language=Solidity, caption={Block Number Weakness in PoW},linerange={4-17}, label={lst:number_weakness}]{../dev/contracts/blocknumbertimelock.sol}

\subsection{Proof-of-Stake Block Number}
In PoS, just like in PoW, the block number increases by one with each new
block. In PoS, the block intervals are for 99.05\% at 12 seconds. Consequently,
it might appear that using the block number as a time proxy in PoS is a viable
option. However, it's not advisable, because validators go offline from time to
time. This causes the block time to increase and therefore leads to an
inaccuracy of the calculation of time based on the block number. \\
% (like the one in Listing \ref{lst:number_weakness})
A contract initially deployed under PoW would
become impractical or malfunction in a PoS environment due to the inherent differences in the block times.

\subsection{Conclusion}
The transition from PoW to PoS has significantly stabilized block times,
particularly over short periods. However, it's important to note that using
block.number as a proxy for time is still not advisable, as the fixed value
with 
12 seconds per slot might change in the future. In earlier PoS designs, the
slot time was set at 6 seconds \cite{block_time_6_to_12_sec}. There have
also been proposals to adjust the slot time to 8 seconds
\cite{proposed_block_time_8_seconds}. This indicates that for shorter
durations, block times under PoS are markedly more stable than those under
PoW. Conversely, for longer timeframes, this stability is less assured due
to the potential for changes in slot time.


\section{Conclusion: Using block values for time under PoW vs PoS}



In PoW-based Ethereum, block times were probabilistic and influenced by mining difficulty.
As a result, using block values as a time proxy in PoW was not secure.
However, since Ethereum's transition to PoS, this is no longer the case.
PoS-based Ethereum has fixed block times of 12 seconds or multiples thereof.
Our research has shown that block times can no longer be influenced from external sources.
Therefore, using the block.timestamp value as a time proxy is now safe in PoS.

Using the block.number as a proxy for time is still not advisable. Block times 
can become a multiple of 12s and it is also not guaranteed that the block intervals will
never change again in the future.
