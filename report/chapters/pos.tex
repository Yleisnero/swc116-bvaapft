\section{Proof-of-stake}

Since the Merge on September 15, 2022 Ethereum uses a proof-of-stake (PoS)
consensus protocol. In PoS Ethereum new blocks are proposed by validators,
which then other randomly selected validators have to vote about its validity. 
In order to run a validator 32 ETH have to be deposited in a deposit
contract. Validators are incentivized to act honestly by threatening to destroy
some or all of the staked ETH if they act dishonestly.

Proof-of-stake Ethereum introduced slots and epoch. Each slot has a time frame
of 12 seconds and an epoch lasts 32 slots
\cite{seconds-per-slot-mainnet}\cite{seconds-per-slot-mainnet-doc}. For each
slot one random validator is selected to propose a new block to the network.
When creating a new block validators have to follow the specification or blocks
will get rejected.

\subsection{Block Timestamp}

The PoS Ethereum consensus specification defines how to calculate the timestamp
at a specific slot with the function \textit{compute\_timestmap\_at\_slot}
\cite{compute-timestamp-at-slot}. It calculates the timestamp with the following formula:

\begin{equation}
genesis\_time + slots\_since\_genesis *
seconds\_per\_slot
\end{equation}

The Ethereum mainnet configuration sets the
value of \textit{seconds\_per\_slot} to 12 seconds
\cite{seconds-per-slot-mainnet} \cite{seconds-per-slot-mainnet-doc}.


Each validator calculates the block timestamp the same way and checks if
timestamp in the block matches the calculated timestamp, if not the block will
get rejected \cite{process-execution-payload}. Each slot and hence each block
has a predetermined timestamp and is therefore impossible to tamper with but
also much more easier to predict. If the selected validator to propose a new
block is offline the slot remains empty \cite{validator-offline}. The following
blocks timestamp has the timestamp of the following slot (i.e. a multiple of
\textit{SECONDS\_PER\_SLOT}). In that case blocks will be 24 seconds apart. If
the next proposed validator is also offline the block time will be 36 seconds
apart etc.

\begin{figure}
  \centering
  \includegraphics[width=1\textwidth]{../block\_time\_analysis/time\_difference\_bar.png}
  \caption{Block times since the merge.}
  \label{fig:block_time_analysis}
\end{figure}

Analysing blockchain data (see Figure \ref{fig:block_time_analysis}) after the
merge (block number $> 15537393$) shows that 99.05 \% of all blocks have a
block time of 12 seconds.


\subsection{Block Time under Pow-Ethereum vs Pos-Ethereum}
\subsection{Block Time}
