
\section{Block Time}

The block time is the time difference between two subsequent blocks i.e.
current block timestamp - parent block timestamp.

\subsection{Block Time under Proof-of-Work}

\subsubsection{Proof-of-Work}

In PoW based systems use an difficutly adjustment algorithm to determine the
block time. If the block time was too small ($<10s$) the difficulty got
increased, and if the block time was too high ($>20s$) the difficulty got
decreased \cite{eip-2}. Hence the block time could vary from block to block.

\subsubsection{Block Time}

Furthermore the only restriction which miners had to follow when setting the 
block timestamp value was the yellow paper. Which defined that the that
the timestamp of the new block has to be greater than the timestamp of the
parent block \cite{ethyellowpaper2023}. Hence miners could gain an unfair
advantage by maniupulating the timestamp of their mined block (todo add
references).

In order to prevent miners from setting the value too far into the future some
of the most used Ethereum implementations like Geth (go-ethereum) rejected
blocks which timestamps are more than 15 seconds in the future
\cite{go-ethereum-15-sek-limit}. 

Hence when the PoW consensus mechanism was used it was possible for the miner
to tamper with the timestamp and change it up to plus 15 seconds. This
behaviour let to the "15-second
Rule"\footnote{\url{https://consensys.github.io/smart-contract-best-practices/development-recommendations/solidity-specific/timestamp-dependence/}},
which is a security best practice which recommends that the usage of the block
timestamp is safe if the time dependent event can vary by up to 15 seconds.



%The mining difficulty in those systems can
%vary and therefore the block time can also vary \cite{eth_blocks}. 
%
%Block times in PoW depended on how long a miner needed to brute-force a hash.
%It is possible to determine how long it takes to guess the right hash in
%average. But the actual time can vary. Also this hash had a certain difficulty
%which influences how long it takes to guess it correctly. This difficulty
%changed because of forks and because of the difficulty bomb. In conclusion this
%means that block times in PoW can vary quite much and are only 14 seconds in
%average.

\subsection{Block Time under Proof-of-Stake}

\subsubsection{Proof-of-Stake}

Since the Merge on September 15, 2022 Ethereum uses a proof-of-stake (PoS)
consensus protocol. In PoS Ethereum new blocks are proposed by validators,
which then other randomly selected validators have to vote about its validity. 
In order to run a validator 32 ETH have to be deposited in a deposit
contract. Validators are incentivized to act honestly by threatening to destroy
some or all of the staked ETH if they act dishonestly.

Proof-of-stake Ethereum introduced slots and epoch. Each slot has a time frame
of 12 seconds and an epoch lasts 32 slots
\cite{seconds-per-slot-mainnet}\cite{seconds-per-slot-mainnet-doc}. For each
slot one random validator is selected to propose a new block to the network.
When creating a new block validators have to follow the specification or blocks
will get rejected.


\subsubsection{Block time}

The PoS Ethereum consensus specification defines how to calculate the timestamp
at a specific slot with the function \textit{compute\_timestmap\_at\_slot}
\cite{compute-timestamp-at-slot}. It calculates the timestamp with the
following formula:

\begin{equation}
genesis\_time + slots\_since\_genesis *
seconds\_per\_slot
\end{equation}

The Ethereum mainnet configuration sets the
value of \textit{seconds\_per\_slot} to 12 seconds
\cite{seconds-per-slot-mainnet} \cite{seconds-per-slot-mainnet-doc}. The mainet
beacon chains value for \textit{genesis\_time} is $1606824023$, which is the
timestamp when the the original PoS beachone chain was launched on December 01
2020.

Each validator calculates the block timestamp the same way and checks if
timestamp in the block matches the calculated timestamp, if not the block will
get rejected \cite{process-execution-payload}. Each slot and hence each block
has a predetermined timestamp and is therefore impossible to tamper with but
also much more easier to predict. If the selected validator to propose a new
block is offline the slot remains empty \cite{validator-offline}. The following
blocks timestamp has the timestamp of the following slot (i.e. a multiple of
\textit{SECONDS\_PER\_SLOT}). In that case blocks will be 24 seconds apart. If
the next proposed validator is also offline the block time will be 36 seconds
apart etc.

\begin{figure}[H]
  \centering
  \includegraphics[width=1\textwidth]{../block\_time\_analysis/time\_difference\_bar.png}
  \caption{Block times since the merge.}
  \label{fig:block_time_analysis}
\end{figure}

Analysing blockchain data (see Figure \ref{fig:block_time_analysis}) after the
merge (block number $> 15537393$) shows that 99.05 \% of all blocks have a
block time of 12 seconds.


\section{Block Time under Proof-of-Work vs. Proof-of-Stake Ethereum}
\subsection{Block Time}
