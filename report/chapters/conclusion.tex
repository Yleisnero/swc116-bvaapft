\section{Conclusion}

Under PoW miners were able to manipulate block timestamps for a up 15 seconds.
Consequently, utilizing the block timestamp for time-dependent events was
secure only if the contract's time-dependent events could tolerate a variation
of up to 15 seconds while still maintaining their integrity. Since PoS
timestamp manipulation is impossible and therefore secure to use since miners
can not gain an unfair advantage by modifing the timestamp.

In PoW Ethereum, block times were probabilistic since they are influenced by
the mining difficulty. In contrast, block times under PoS are significantly more
consistent, mostly 12 seconds, compared to those in PoW. Therefore, in PoS,
using block numbers as a measure of time is generally more accurate than in
PoW, especially for shorter durations (possibly less than a year). However,
it's important to note that the fixed block time (slot time) in PoS is not
guaranteed to remain unchanged indefinitely. Therefore, using block numbers as
a time proxy is generally not advisable.


%In PoW-based Ethereum, block times were probabilistic and influenced by mining difficulty.
%As a result, using block values as a time proxy in PoW was not secure.
%However, since Ethereum's transition to PoS, this is no longer the case.
%PoS-based Ethereum has fixed block times of 12 seconds or multiples thereof.
%Our research has shown that block times can no longer be influenced from external sources.
%Therefore, using the block.timestamp value as a time proxy is now safe in PoS.
%
%Using the block.number as a proxy for time is still not advisable. Block times 
%can become a multiple of 12s and it is also not guaranteed that the block intervals will
%never change again in the future.

