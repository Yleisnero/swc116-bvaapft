\documentclass{article}
\usepackage{solidity}
\usepackage{hyperref}
\title{SWC-116: Block values as a proxy for time}
\author{Lorenz Kofler and Jonas Fischer}
\date{13 October 2023}
\bibliographystyle{IEEEtran}
\begin{document}
\maketitle
\tableofcontents
\newpage

\section{Introduction}
This report deals with a weakness in Solidity smart contracts, which is called
"block values as a proxy for time". This weakness is described in the Smart
Contract Weakness Classification (SWC) registry under the number 116. \newline
The weakness comes from the attempt of developers to introduce time depended
functionality in their smart contracts, by reading the values block.timestamp
or block.number \cite{swc116}. \newline

\section{Background}
\subsection{block.timestamp}
Because Ethereum is decentralized, the block timestamp is controlled by the
miners. In the Ethereum Yellow-Paper is no restriction how the miners have to set the timestamp. It
only needs to be higher then the timestamp of the previous block \cite{Conkas2021}.
\begin{equation} \label{eq:1}
H_s > P(H)_{H_s}
\end{equation}
The timestamp of the block $H_s$ in \ref{eq:1} must be greater then the timestamp of the previous block $P(H)_{H_s}$ \cite{ethyellowpaper2023}.
Nevertheless, in implementations of Ethereum, like Geth and Parity there is a restriction. The timestamp of the new block
must at most 15 seconds in the future \cite{Conkas2021}. \newline

\begin{lstlisting}[language=go, caption="The restriction for the timestamp in Geth. Source: consensus/ethash/consensus.go \cite{timestamp_code}"]
func (ethash *Ethash) verifyHeader(chain consensus.ChainHeaderReader, header, parent *types.Header, uncle bool, unixNow int64) error {
  ...
	// Verify the header's timestamp
		if header.Time > uint64(unixNow+allowedFutureBlockTimeSeconds) {
			return consensus.ErrFutureBlock
		}
	if header.Time <= parent.Time {
		return errOlderBlockTime
	}
  ...
}
\end{lstlisting}

\subsection{block.number}

\subsection{Consistency of block time since Proof of Stake}
In Proof of Work based systems block times are probabilistic, because they depend on the mining difficulty.
The mining difficulty in those systems can vary and therefore the block time can also vary \cite{eth_blocks}. \newline
In contrast Proof of Stake based systems have a constant block time.
Before the Paris update of Ethereum, Ethereum was based on Proof of Stake. The so called Merge introduced Proof of Stake for Ethereum
on the 15.09.2022 \cite{eth_history}. The current version of Ethereum has a constant block time of 12 seconds \cite{eth_blocks}.

\section{Variants of the weakness}

\subsection{Block - Timestamp}
The first variant of the block values as a proxy for time weakness occurs if the value block.timestamp is used in a smart contract.
If developers try to to use the timestamp value as a source for an exact time this could leak to errors or false behaviors, because 
the timestamp could be set wrong by the miner of the block.

\subsection{Block - Number}
Another variant of can happen, when the number value of the block is used. Developers could try to use the block number as a proxy for time,
by dividing it by fourteen, because they assume that the time between two blocks is always around fourteen seconds.

\subsection{Non weaknesses}

\section{Examples} 
\subsection{Example1 - Game} \label{ex:1}
\begin{solidity} 
    function play() public {
        require(block.timestamp > 1521763200 && neverPlayed == true);
        neverPlayed = false;
        msg.sender.transfer(1500 ether);
    }
\end{solidity}
Source: NCC Group - Time manipulation \cite{DASP2018}

\subsection{Example2 - TimeLock} \label{ex:2}
\begin{solidity} 
    contract TimeLock {
		...
		function lockEth(uint _time, uint _amount) public payable {
			...
			users[msg.sender].unlockBlock = block.number + (_time / 14);
			...
		}

    function withdraw() public {
		...
        require(block.number >= users[msg.sender].unlockBlock, 'lock period not over');
		...
    }
}
\end{solidity}
Source: SWC116 \cite{swc116}

\section{Consequences}
If one of the described weaknesses occurs in a smart contract it can happen, that e.g. users of the smart contract can not access
their ethereum or other information stored in the contract as shown in Example \ref{ex:2}. \newline
Also it can happen that a miner is gaining an unfair advantage if he is able to set the timestamp of his mined block into the future,
such that he is able to access resources earlier then other user as shown in Example \ref{ex:1}.

\section{Tools}
\section{Exploits}

\section{Infos}
There is neither a lower nor an upper bound on the number of pages. Assume that
the reader is familiar with Ethereum and Smart Contracts, and just write what
is necessary. It is a technical report, not an essay, so stick to the facts. As
a sign of professionalism, proofread the paper and eliminate spelling and
grammatical errors. You may use chatGPT or a similar tool to polish the paper,
but take care that it does not introduce nonsense.

Write the paper in English. Typeset it in \LaTeX. The document class
\verb"article" is fine, but you may use fancier formatting if you prefer. See
the source of this document for an example.

The citation information for the papers in the TUWEL course is provided
in the file \verb"seminar.bib". Add any further papers that you want to cite.
Cite only the papers that you really need.
Currently, \verb"seminar.bib" contains the following references.
\nocite{*}

\bibliography{seminar}
\end{document}
