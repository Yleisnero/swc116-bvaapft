\documentclass{article}
\usepackage{pgfplots}
\usepackage{pgfplotstable}
\usepackage{solidity}
\usepackage{hyperref}
\usepackage{filecontents}

\pgfplotstableread[col sep=comma]{blocktime.csv}\datatable

\usepgfplotslibrary{external}
\tikzexternalize

\title{SWC-116: Block values as a proxy for time}
\author{Lorenz Kofler and Jonas Fischer}
\date{13 October 2023}
\bibliographystyle{IEEEtran}
\begin{document}
\maketitle
\tableofcontents
\newpage

\section{Introduction}
This report deals with a weakness in Solidity smart contracts, which is called
"block values as a proxy for time". This weakness is described in the Smart
Contract Weakness Classification (SWC) registry under the number 116. \newline
The weakness comes from the attempt of developers to introduce time depended
functionality in their smart contracts, by reading the values block.timestamp
or block.number \cite{swc116}. \newline

\section{Background}
\subsection{block.timestamp}
\section{Block timestamp}

\subsection{Background} 
The miner is responsible for setting the timestamp within a block. The yellow
paper defines which rules this value has to follow. But depending on the
specific consensus mechanism there are also consensous specification and
software imposed restrictions for the timestamp value. 

\paragraph{Yellow Paper}
Pre-merge the only restriction was the yellow paper, which set that the that
the timestamp of the new block has to be greater than the timestamp of the
parent block \cite{ethyellowpaper2023}.

\paragraph{Software Implementation - Proof-of-work}
The currently most used ethereum implementation Geth rejects blocks which
timestamps are more than 15 seconds in the future
\cite{go-ethereum-15-sek-limit}. Hence when the proof-of-work consensus
mechanism was used it was possible for the miner to tamper
with the timestamp and change it up to plus 15 seconds.

\paragraph{Software Implementation - Proof-of-stake}
Proof-of-stake ethereum introduced slots, where each slot has a time frame of
12 seconds \cite{seconds-per-slot-mainnet}\cite{seconds-per-slot-mainnet-doc}.
For each slot one random validator is selected to propose a block, which other
random validatores have vote about its validity. The specification defines
how to calculate the timestamp at a specific slot with the function
\textit{compute\_timestmap\_at\_slot} \cite{compute-timestamp-at-slot}. The
timestamp is calculated with $genesis\_time + slots\_since\_genesis *
SECONDS\_PER\_SLOT$. The function \textit{process\_execution\_payload} then
checks if the blocks timestamp equals the calculated timestamp at that slot
\cite{process-execution-payload}. Hence each slot has a predefined timestamp
and it is therefore impossible to tamper with but also much more easier to
predict.
% \begin{lstlisting}[language=python, caption=proof-of-stake consensous
% specification\cite{compute-timestamp-at-slot}]
% def compute_timestamp_at_slot(state: beaconstate, slot: slot) -> uint64:
%     slots_since_genesis = slot - genesis_slot
%     return uint64(state.genesis_time + slots_since_genesis * seconds_per_slot)
% \end{lstlisting}



% \paragraph{Yellow Paper}
% Miners set the block timestamp during the mining process (source), in
% that process miners have to follow only the restriction of the Ethereum
% Yellow-Paper, which is the following:
% 
% We define $H_s$ is the timestamp of Block $H$, $P(H)$ is the parent
% block of block $H$. The Yellow-Paper defines the following relation for a valid
% block timestamp \cite{ethyellowpaper2023}.
% 
% This definition essentially means the timestamp of the block $H_s$ in
% \ref{eq:1} must be greater then the timestamp of the previous (parent) block
% $P(H)_{H_s}$ \cite{ethyellowpaper2023}.

%\paragraph{Software Implementation}
%
%\begin{lstlisting}[language=go, caption=The restriction for the timestamp in Geth. Source: \textit{consensus/ethash/consensus.go} \cite{timestamp_code}]
%func (ethash *Ethash) verifyHeader(chain consensus.ChainHeaderReader, header, parent *types.Header, uncle bool, unixNow int64) error {
%  ...
%    // Verify the header's timestamp
%        if header.Time > uint64(unixNow+allowedFutureBlockTimeSeconds) {
%            return consensus.ErrFutureBlock
%        }
%    if header.Time <= parent.Time {
%        return errOlderBlockTime
%    }
%  ...
%}
%\end{lstlisting}
%
%\paragraph{Issues}

\begin{itemize}
\item Do not use block.timestamp as time locks, since developer should follow the yellow paper, it is not sure that software implements the restrictions.
\item Do not use block.timestamp as a source of randomness.
\end{itemize}

\subsection{Examples}
\subsubsection{Example containing weakness}
\begin{solidity}
contract Firework {
    function startFirework() public {
        // 01.01.2024 00:00:00 GMT+0100
        require(block.timestamp > 1704063600);
    }
}
\end{solidity}

\subsubsection{Example without a weakness}
\begin{solidity}
contract Game {
    uint expiry;

    constructor(uint expiryTimestamp) public {
        expiry = expiryTimestamp;
    }

    function play() public {
        // Safe to use because block timestamp can not be modified backwards
        require(block.timestamp < expiry);
    }
}
\end{solidity}
    
\subsection{Consequences}
A miner is gaining an unfair advantage if he is able to set the timestamp
of his mined block into the future, such that he is able to access
resources earlier then other users or trigger events before they are meant to happen.

Because Ethereum is decentralized, the block timestamp is controlled by the
miners. In the Ethereum Yellow-Paper is no restriction how the miners have to set the timestamp. It
only needs to be higher then the timestamp of the previous block \cite{Conkas2021}.
\begin{equation} \label{eq:2}
H_s > P(H)_{H_s}
\end{equation}
The timestamp of the block $H_s$ in \ref{eq:1} must be greater then the timestamp of the previous block $P(H)_{H_s}$ \cite{ethyellowpaper2023}.
Nevertheless, in implementations of Ethereum, like Geth and Parity there is a restriction. The timestamp of the new block
must at most 15 seconds in the future \cite{Conkas2021}. \newline

\begin{lstlisting}[language=go, caption="The restriction for the timestamp in Geth. Source: consensus/ethash/consensus.go \cite{timestamp_code}"]
func (ethash *Ethash) verifyHeader(chain consensus.ChainHeaderReader, header, parent *types.Header, uncle bool, unixNow int64) error {
  ...
	// Verify the header's timestamp
		if header.Time > uint64(unixNow+allowedFutureBlockTimeSeconds) {
			return consensus.ErrFutureBlock
		}
	if header.Time <= parent.Time {
		return errOlderBlockTime
	}
  ...
}
\end{lstlisting}

\subsection{block.number}
\section{Block number}
\subsection{Proof-of-Work}
The yellow paper of Ethereum states that each block has
an integer value as a block number. The block number
of a specific block must be exactly one unit higher than the block
number of the previous block \cite{ethyellowpaper2023}.
Smart contract developers attempted to utilize the block number to calculate
the current time, operating under the assumption that the time interval between
two blocks always remains constant at fourteen seconds. Theoretically, one could
calculate time differences between blocks by multiplying the block numbers by fourteen 
and subtracting them from each other.
However, this approach proves impractical in reality due to the variable
intervals between blocks, as explained earlier. Moreover, these block intervals
can change unexpectedly, for instance, when the difficulty bomb was introduced
or during a chain fork \cite{swc116}.
As a result the block number should not be used as a proxy for time.
%An example for this misusage can be seen in the Listing \ref{lst:number_weakness}.
%There the block number is used to lock a function of a contract for a specific time.

%\lstinputlisting[language=Solidity, caption={Block Number Weakness in PoW},linerange={4-17}, label={lst:number_weakness}]{../dev/contracts/blocknumbertimelock.sol}

\subsection{Proof-of-Stake}
In PoS, just like in PoW, the block number increases by one with each new
block. In PoS, the block intervals are for 99.05\% of times at 12 seconds. Consequently,
it might appear that using the block number as a time proxy in PoS is a viable
option. However, it's not advisable, because validators go offline from time to
time. This causes the block time to increase and therefore leads to an
inaccuracy of the calculation of time based on the block number. \\
% (like the one in Listing \ref{lst:number_weakness})
A contract initially deployed under PoW would
become impractical or malfunction in a PoS environment due to the inherent differences in the block times.

\subsection{Summary}
The transition from PoW to PoS has significantly stabilized block times,
particularly over short periods. However, it's important to note that using
the block number as a proxy for time is still not advisable, as the fixed value
with 
12 seconds per slot might change in the future. In earlier PoS designs, the
slot time was set at 6 seconds \cite{block_time_6_to_12_sec}. There have
also been proposals to adjust the slot time to 8 seconds
\cite{proposed_block_time_8_seconds}. This indicates that for shorter
durations, block times under PoS are markedly more stable than those under
PoW. Conversely, for longer timeframes, this stability is less assured due
to the potential for changes in slot time.



\begin{tikzpicture}
    \begin{axis}[
      height=10cm,
      ymin = 0,
      xtick=data,
      xticklabels from table={\datatable}{Date(UTC)},
      x tick label style={font=\normalsize, rotate=90, anchor=east},
      ylabel={Average Block Time (s)}]
      \addplot table [x expr=\coordindex, y={AverageBlockTime}]{\datatable};
    \end{axis}
\end{tikzpicture}


\subsection{Consistency of block time since Proof of Stake}
In Proof of Work based systems block times are probabilistic, because they depend on the mining difficulty.
The mining difficulty in those systems can vary and therefore the block time can also vary \cite{eth_blocks}. \newline
In contrast Proof of Stake based systems have a constant block time.
Before the Paris update of Ethereum, Ethereum was based on Proof of Stake. The so called Merge introduced Proof of Stake for Ethereum
on the 15.09.2022 \cite{eth_history}. The current version of Ethereum has a constant block time of 12 seconds \cite{eth_blocks}.

\section{Variants of the weakness}

\subsection{Block - Timestamp}
The first variant of the block values as a proxy for time weakness occurs if the value block.timestamp is used in a smart contract.
If developers try to to use the timestamp value as a source for an exact time this could leak to errors or false behaviors, because 
the timestamp could be set wrong by the miner of the block.

\subsection{Block - Number}
Another variant of can happen, when the number value of the block is used. Developers could try to use the block number as a proxy for time,
by dividing it by fourteen, because they assume that the time between two blocks is always around fourteen seconds.

\subsection{Non weaknesses}

\section{Examples} 
\subsection{Example1 - Game} \label{ex:1}
\begin{solidity} 
    function play() public {
        require(block.timestamp > 1521763200 && neverPlayed == true);
        neverPlayed = false;
        msg.sender.transfer(1500 ether);
    }
\end{solidity}
Source: NCC Group - Time manipulation \cite{DASP2018}

\subsection{Example2 - TimeLock} \label{ex:2}
\begin{solidity} 
    contract TimeLock {
		...
		function lockEth(uint _time, uint _amount) public payable {
			...
			users[msg.sender].unlockBlock = block.number + (_time / 14);
			...
		}

    function withdraw() public {
		...
        require(block.number >= users[msg.sender].unlockBlock, 'lock period not over');
		...
    }
}
\end{solidity}
Source: SWC116 \cite{swc116}

\section{Consequences}
If one of the described weaknesses occurs in a smart contract it can happen, that e.g. users of the smart contract can not access
their ethereum or other information stored in the contract as shown in Example \ref{ex:2}. \newline
Also it can happen that a miner is gaining an unfair advantage if he is able to set the timestamp of his mined block into the future,
such that he is able to access resources earlier then other user as shown in Example \ref{ex:1}.

\section{Tools}
\section{Exploits}

\section{Infos}
There is neither a lower nor an upper bound on the number of pages. Assume that
the reader is familiar with Ethereum and Smart Contracts, and just write what
is necessary. It is a technical report, not an essay, so stick to the facts. As
a sign of professionalism, proofread the paper and eliminate spelling and
grammatical errors. You may use chatGPT or a similar tool to polish the paper,
but take care that it does not introduce nonsense.

Write the paper in English. Typeset it in \LaTeX. The document class
\verb"article" is fine, but you may use fancier formatting if you prefer. See
the source of this document for an example.

The citation information for the papers in the TUWEL course is provided
in the file \verb"seminar.bib". Add any further papers that you want to cite.
Cite only the papers that you really need.
Currently, \verb"seminar.bib" contains the following references.
\nocite{*}

\bibliography{seminar}
\end{document}
